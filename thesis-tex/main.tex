\documentclass[en]  {pracamgr}
%include standalone forest, tikz figures
\usepackage{standalone}
\usepackage{biblatex}
% recommended while using biblatex with babel
\usepackage{csquotes}
\usepackage{amsmath}
% make references clickable, hide boxes around links
\usepackage[hidelinks, pdfpagelabels]{hyperref}
\addbibresource{references.bib}

% https://tex.stackexchange.com/a/404990
% https://mirror.aarnet.edu.au/pub/CTAN/macros/latex/contrib/hyperref/doc/hyperref-doc.html#x1-140004
\addto\extrasenglish{%
  \def\chapterautorefname{Chapter}%
}
\addto\extrasenglish{%
  \def\chaptername{Chapter}%
}

\addto\extrasenglish{%
  \def\sectionautorefname{Section}%
}
\addto\extrasenglish{%
  \def\sectionname{Section}%
}
\usepackage{subcaption}
\usepackage{listings}
\lstset{
  basicstyle=\ttfamily,
  columns=fullflexible,
  frame=single,
  breaklines=true,
  literate={ą}{{\k a}}1
  	  {Ą}{{\k A}}1
           {ż}{{\. z}}1
           {Ż}{{\. Z}}1
           {ź}{{\' z}}1
           {Ź}{{\' Z}}1
           {ć}{{\' c}}1
           {Ć}{{\' C}}1
           {ę}{{\k e}}1
           {Ę}{{\k E}}1
           {ó}{{\' o}}1
           {Ó}{{\' O}}1
           {ń}{{\' n}}1
           {Ń}{{\' N}}1
           {ś}{{\' s}}1
           {Ś}{{\' S}}1
           {ł}{{\l}}1
           {Ł}{{\L}}1
}
\usepackage{url}
\usepackage{xr}
\usepackage{tabularx}
\usepackage{booktabs}
\usepackage{graphicx}
\usepackage{forest}
\usepackage{tikz-qtree}
% https://tex.stackexchange.com/a/18927


\usepackage{pdflscape}

\usepackage[titletoc]{appendix}
\usepackage{amsthm}


\autor{Paweł Balawender}{429141}

\title{Practical programming languages capturing complexity classes}
\titlepl{Praktyczne języki programowania wyrażające klasy złożoności}
% \tytulang{An implementation of a difference blabalizer based on the theory of $\sigma$ -- $\rho$ phetors}

%kierunek: 
% - matematyka, informacyka, ...
% - Mathematics, Computer Science, ...
\kierunek{Computer Science}

% informatyka - nie okreslamy zakresu (opcja zakomentowana)
% matematyka - zakres moze pozostac nieokreslony,
% a jesli ma byc okreslony dla pracy mgr,
% to przyjmuje jedna z wartosci:
% {metod matematycznych w finansach}
% {metod matematycznych w ubezpieczeniach}
% {matematyki stosowanej}
% {nauczania matematyki}
% Dla pracy licencjackiej mamy natomiast
% mozliwosc wpisania takiej wartosci zakresu:
% {Jednoczesnych Studiow Ekonomiczno--Matematycznych}

% \zakres{Tu wpisac, jesli trzeba, jedna z opcji podanych wyzej}

% Praca wykonana pod kierunkiem:
% (podać tytuł/stopień imię i nazwisko opiekuna
% Instytut
% ew. Wydział ew. Uczelnia (jeżeli nie MIM UW))
\opiekun{dr hab. Paweł Parys, prof. UW\\
  Institute of Informatics, University of Warsaw\\
  }

% miesiąc i~rok:
\date{July~2025}

%Podać dziedzinę wg klasyfikacji Socrates-Erasmus:
\dziedzina{ 
%11.0 Matematyka, Informatyka:\\ 
%11.1 Matematyka\\ 
%11.2 Statystyka\\ 
11.3 Computer Science\\ 
%11.4 Sztuczna inteligencja\\ 
%11.5 Nauki aktuarialne\\
%11.9 Inne nauki matematyczne i informatyczne
}

%Klasyfikacja tematyczna wedlug AMS (matematyka) lub ACM (informatyka)
\klasyfikacja{F. Theory of computation\\
  F.3. Logics and meanings of programs\\
  F.3.3. Studies of program constructs}

% Słowa kluczowe:
\keywords{Implicit Computational Complexity, Safe recursion, Logspace, Bellantoni and Cook}

% Tu jest dobre miejsce na Twoje własne makra i~środowiska:
% \newtheorem{defi}{Definicja}[section]
\newcommand{\bcem}{\ensuremath{\text{BC}_{\varepsilon}^-}}
\newtheorem{thm}{Theorem}
\newtheorem{lemma}[thm]{Lemma}
\newtheorem{prop}{Proposition}

% koniec definicji

\begin{document}
\hypersetup{pageanchor=false}
\maketitle

\begin{abstract}
  In this work, I study what features does it make sense to add to a programming language
  from the computational complexity perspective. Specifically, I focus on how these features can be used to capture various complexity classes, such as LOGSPACE, PTIME, and PSPACE. By analyzing the expressiveness and limitations of these features, I aim to provide insights into the design of programming languages that are both practical and theoretically sound.
\end{abstract}

\hypersetup{pageanchor=true}
\tableofcontents
%\listoffigures
%\listoftables

\chapter*{Neergaard's paper from 2004}
\addcontentsline{toc}{chapter}{Neergaard}

Let's define Neergaard's LOGSPACE algebra: BC$_N$ is the smallest class
of functions containing the initial functions i-v and closed under vi, vii:
\begin{enumerate}
  \item i. (Constant) 0 (a zero-ary function)
  \item ii. (Projection) $\pi_j^{n, m}(x_1, \dots, x_n : x_{n+1}, \dots, x_{n+m}) = x_j$ for $1 \leq j \leq n+m$
  \item iii. (Successors) $s_i(:a) = 2a + 1 = ai$, for $i \in \{0, 1\}$
  \item iv. (Predecessor) $p(: 0) = 0$, $p(: ai) = a$
  \item v. (Conditional) $C(: a, b, c) = b$ if $a \mod 2 = 1$, $c$ otherwise
  \item vi. (Safe affine recursion) Define the new function $f$ by, for $i \in \{0, 1\}$:
        \begin{align*}
          f(0, x: a)  & = g(x:a)                 \\
          f(yi, x: a) & = h_i(y, x: f(y, x : a))
        \end{align*}
        where $h_i, g$ are in BC$_N$.
  \item vii. (Safe composition) Define the new function $f$ by
        \begin{align*}
          f(x : a) = h(r_1(x:), r_2(x:), \dots : t_1(x : a_1), t_2(x : a_2), \dots)
        \end{align*}
        where $h, r_1, r_2, \dots, t_1, t_2, \dots$ are in BC$_N$, and $a_1, a_2, \dots$ are partitions of $a$.
\end{enumerate}

In his work from~2004~\cite{10.1007/978-3-540-30477-7_21}, Neergaard says that: 
\begin{prop}
  Let $m$ and $n$ be numbers in binary. Right shift $\text{shift}^R(m:n)$ of $m$ by $|n|$ and selection of bit $|n|$
from m are definable in \bcem
\end{prop}

Before talking about the above proposition, let's consider another function, cond'.

\includestandalone[mode=buildnew, width=\textwidth]{neergaard-ast}

\begin{lemma}
  Function $\text{cond'}(:x, y, z) = y$ if $x = \varepsilon$, $z$ otherwise, is not definable in \bcem
\end{lemma}
\begin{proof}
  First, notice that a recursion node cannot occur anywhere in the syntax tree of function cond', because then it would need to take at least one normal argument, which it does not. So it has to be constructed from compositions of base functions and thus can be represented by a simple tree, like the function in the figure. The nodes x, y, z would be substituted by the representation of the actual argument, e.g. if we passed 0 as x to the function, we would have a O(:) node 
  instead of x in the tree. Since we don't use recursion, the AST precisely represents the computation tree as well. Consider the path from every safe argument to the root of the tree (which is the result). Every safe argument will either at some point occur as the first argument of some C node (conditional), or not. \\
  Towards obtaining a contradiction, assume that function cond' belongs to the class $\bcem$ and consider its AST.
  Consider the path from the node representing argument x (so the argument we condition on in cond') to the root.
If x at some point is the first argument of some C node, the AST determines the highest possible bit we can check in this node. So if, after taking the argument x, we immediately condition on it, the highest bit tested is bit 0, which we will only check if x is not empty. If before condition we have 2 p nodes, we will condition on bit 2, etc. This number is constant for the whole function and does not depend on values of arguments passed. So we modify a finite prefix of x, then condition on its least significant bit. It means that there exist $k$ such that $cond'(:x, y, z) = cond'(1(0^k)x, y, z$ -- the C node identifies all even numbers, so it cannot tell an empty string from a one with many least significant zeros, and then a one on MSB. But clearly this property of cond' is against its definition, as cond'(:0,y, z) = y, while cond'(:100000000..., y, z) = z.
  Now assume that x never ends up as the first argument of any C node. Then either x gets 'destroyed' on its path, either not being selected by a pi node or by a C node, or it actually ends up in the result with a modified finite prefix of length K. Then clearly the result will not be correct for both $cond'(:1(0^k)x, y, z)$ and $cond'(:11(0^k)x, y, z)$ at the same time.
\end{proof}

and now:
\begin{thm}
  Let $m$ and $n$ be numbers in binary. Right shift $\text{shift}^R(m:n)$ of $m$ by $|n|$ is not definable in \bcem.
\end{thm}
\begin{proof}
  By contradiction, assume it is definable. Then $\text{cond'}(:x, y, z) = C(: \text{shift}^R(1 : x), y, z)$, but cond' 
  is not definable.
\end{proof}

This is important that

\appendix

\chapter{Przykładowe wyniki blabalizy
  (ze~współczynnikami~$\sigma$-$\rho$)}

\begin{center}
  \begin{tabular}{lrrrr}
                  & Współczynniki                                           \\
                  & Głombaskiego  & $\rho$     & $\sigma$ & $\sigma$-$\rho$ \\
    $\gamma_{0}$  & 1,331         & 2,01       & 13,42    & 0,01            \\
    $\gamma_{1}$  & 1,331         & 113,01     & 13,42    & 0,01            \\
    $\gamma_{2}$  & 1,332         & 0,01       & 13,42    & 0,01            \\
    $\gamma_{3}$  & 1,331         & 51,01      & 13,42    & 0,01            \\
    $\gamma_{4}$  & 1,332         & 3165,01    & 13,42    & 0,01            \\
    $\gamma_{5}$  & 1,331         & 1,01       & 13,42    & 0,01            \\
    $\gamma_{6}$  & 1,330         & 0,01       & 13,42    & 0,01            \\
    $\gamma_{7}$  & 1,331         & 16435,01   & 13,42    & 0,01            \\
    $\gamma_{8}$  & 1,332         & 865336,01  & 13,42    & 0,01            \\
    $\gamma_{9}$  & 1,331         & 34,01      & 13,42    & 0,01            \\
    $\gamma_{10}$ & 1,332         & 7891432,01 & 13,42    & 0,01            \\
    $\gamma_{11}$ & 1,331         & 8913,01    & 13,42    & 0,01            \\
    $\gamma_{12}$ & 1,331         & 13,01      & 13,42    & 0,01            \\
    $\gamma_{13}$ & 1,334         & 789,01     & 13,42    & 0,01            \\
    $\gamma_{14}$ & 1,331         & 4897453,01 & 13,42    & 0,01            \\
    $\gamma_{15}$ & 1,329         & 783591,01  & 13,42    & 0,01            \\
  \end{tabular}
\end{center}

\printbibliography[heading=bibintoc]
% \begin{thebibliography}{99}
% \addcontentsline{toc}{chapter}{Bibliografia}

% \bibitem[Bea65]{beaman} Juliusz Beaman, \textit{Morbidity of the Jolly
%     function}, Mathematica Absurdica, 117 (1965) 338--9.
% \end{thebibliography}

\end{document}


%%% Local Variables:
%%% mode: latex
%%% TeX-master: t
%%% coding: latin-2
%%% End:
