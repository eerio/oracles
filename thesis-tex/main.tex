\documentclass[en]  {pracamgr}
\usepackage{biblatex}
% recommended while using biblatex with babel
\usepackage{csquotes}
\usepackage{amsmath}
% make references clickable, hide boxes around links
\usepackage[hidelinks]{hyperref}
\addbibresource{references.bib}

\autor{Paweł Balawender}{429141}

\title{Practical programming languages capturing complexity classes}
\titlepl{Praktyczne języki programowania wyrażające klasy złożoności}
% \tytulang{An implementation of a difference blabalizer based on the theory of $\sigma$ -- $\rho$ phetors}

%kierunek: 
% - matematyka, informacyka, ...
% - Mathematics, Computer Science, ...
\kierunek{Computer Science}

% informatyka - nie okreslamy zakresu (opcja zakomentowana)
% matematyka - zakres moze pozostac nieokreslony,
% a jesli ma byc okreslony dla pracy mgr,
% to przyjmuje jedna z wartosci:
% {metod matematycznych w finansach}
% {metod matematycznych w ubezpieczeniach}
% {matematyki stosowanej}
% {nauczania matematyki}
% Dla pracy licencjackiej mamy natomiast
% mozliwosc wpisania takiej wartosci zakresu:
% {Jednoczesnych Studiow Ekonomiczno--Matematycznych}

% \zakres{Tu wpisac, jesli trzeba, jedna z opcji podanych wyzej}

% Praca wykonana pod kierunkiem:
% (podać tytuł/stopień imię i nazwisko opiekuna
% Instytut
% ew. Wydział ew. Uczelnia (jeżeli nie MIM UW))
\opiekun{dr hab. Paweł Parys, prof. UW\\
  Instytut Informatyki UW\\
  }

% miesiąc i~rok:
\date{July~2025}

%Podać dziedzinę wg klasyfikacji Socrates-Erasmus:
\dziedzina{ 
%11.0 Matematyka, Informatyka:\\ 
%11.1 Matematyka\\ 
%11.2 Statystyka\\ 
11.3 Informatyka\\ 
%11.4 Sztuczna inteligencja\\ 
%11.5 Nauki aktuarialne\\
%11.9 Inne nauki matematyczne i informatyczne
}

%Klasyfikacja tematyczna wedlug AMS (matematyka) lub ACM (informatyka)
\klasyfikacja{F. Theory of computation\\
  F.3. Logics and meanings of programs\\
  F.3.3. Studies of program constructs}

% Słowa kluczowe:
\keywords{Implicit Computational Complexity, Safe recursion, Logspace, Bellantoni and Cook}

% Tu jest dobre miejsce na Twoje własne makra i~środowiska:
\newtheorem{defi}{Definicja}[section]

% koniec definicji

\begin{document}
\maketitle

%tu idzie streszczenie na strone poczatkowa
\begin{abstract}
  In this work, I study what features does it make sense to add to a programming language
  from the computational complexity perspective.
\end{abstract}

\tableofcontents
%\listoffigures
%\listoftables

\chapter*{Neergaard's paper from 2004}
\addcontentsline{toc}{chapter}{Neergaard}

Let's define Neergaard's LOGSPACE algebra: BC$_N$ is the smallest class
of functions containing the initial functions i-v and closed under vi, vii:
\begin{enumerate}
  \item i. (Constant) 0 (a zero-ary function)
  \item ii. (Projection) $\pi_j^{n, m}(x_1, \dots, x_n : x_{n+1}, \dots, x_{n+m}) = x_j$ for $1 \leq j \leq n+m$
  \item iii. (Successors) $s_i(:a) = 2a + 1 = ai$, for $i \in \{0, 1\}$
  \item iv. (Predecessor) $p(: 0) = 0$, $p(: ai) = a$
  \item v. (Conditional) $C(: a, b, c) = b$ if $a \mod 2 = 1$, $c$ otherwise
  \item vi. (Safe affine recursion) Define the new function $f$ by, for $i \in \{0, 1\}$:
    \begin{align*}
      f(0, x: a) &= g(x:a) \\
      f(yi, x: a) &= h_i(y, x: f(y, x : a))
    \end{align*}
    where $h_i, g$ are in BC$_N$.
  \item vii. (Safe composition) Define the new function $f$ by
    \begin{align*}
      f(x : a) = h(r_1(x:), r_2(x:), \dots : t_1(x : a_1), t_2(x : a_2), \dots)
    \end{align*}
    where $h, r_1, r_2, \dots, t_1, t_2, \dots$ are in BC$_N$, and $a_1, a_2, \dots$ are partitions of $a$.
\end{enumerate}

In his work from~2004~\cite{10.1007/978-3-540-30477-7_21}, Neergaard says:
you can define this! Does it work?




\appendix

\chapter{Przykładowe wyniki blabalizy
    (ze~współczynnikami~$\sigma$-$\rho$)}

\begin{center}
  \begin{tabular}{lrrrr}
    & Współczynniki \\
    & Głombaskiego & $\rho$ & $\sigma$ & $\sigma$-$\rho$\\
    $\gamma_{0}$ & 1,331 & 2,01 & 13,42 & 0,01 \\
    $\gamma_{1}$ & 1,331 & 113,01 & 13,42 & 0,01 \\
    $\gamma_{2}$ & 1,332 & 0,01 & 13,42 & 0,01 \\
    $\gamma_{3}$ & 1,331 & 51,01 & 13,42 & 0,01 \\
    $\gamma_{4}$ & 1,332 & 3165,01 & 13,42 & 0,01 \\
    $\gamma_{5}$ & 1,331 & 1,01 & 13,42 & 0,01 \\
    $\gamma_{6}$ & 1,330 & 0,01 & 13,42 & 0,01 \\
    $\gamma_{7}$ & 1,331 & 16435,01 & 13,42 & 0,01 \\
    $\gamma_{8}$ & 1,332 & 865336,01 & 13,42 & 0,01 \\
    $\gamma_{9}$ & 1,331 & 34,01 & 13,42 & 0,01 \\
    $\gamma_{10}$ & 1,332 & 7891432,01 & 13,42 & 0,01 \\
    $\gamma_{11}$ & 1,331 & 8913,01 & 13,42 & 0,01 \\
    $\gamma_{12}$ & 1,331 & 13,01 & 13,42 & 0,01 \\
    $\gamma_{13}$ & 1,334 & 789,01 & 13,42 & 0,01 \\
    $\gamma_{14}$ & 1,331 & 4897453,01 & 13,42 & 0,01 \\
    $\gamma_{15}$ & 1,329 & 783591,01 & 13,42 & 0,01 \\
  \end{tabular}
\end{center}

\printbibliography[heading=bibintoc]
% \begin{thebibliography}{99}
% \addcontentsline{toc}{chapter}{Bibliografia}

% \bibitem[Bea65]{beaman} Juliusz Beaman, \textit{Morbidity of the Jolly
%     function}, Mathematica Absurdica, 117 (1965) 338--9.
% \end{thebibliography}

\end{document}


%%% Local Variables:
%%% mode: latex
%%% TeX-master: t
%%% coding: latin-2
%%% End:
